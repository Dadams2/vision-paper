% VLDB template version of 2020-08-03 enhances the ACM template, version 1.7.0:
% https://www.acm.org/publications/proceedings-template
% The ACM Latex guide provides further information about the ACM template

\documentclass[sigconf, nonacm]{acmart}

%% The following content must be adapted for the final version
% paper-specific
\newcommand\vldbdoi{XX.XX/XXX.XX}
\newcommand\vldbpages{XXX-XXX}
% issue-specific
\newcommand\vldbvolume{14}
\newcommand\vldbissue{1}
\newcommand\vldbyear{2020}
% should be fine as it is
\newcommand\vldbauthors{\authors}
\newcommand\vldbtitle{\shorttitle} 
% leave empty if no availability url should be set
\newcommand\vldbavailabilityurl{URL_TO_YOUR_ARTIFACTS}
% whether page numbers should be shown or not, use 'plain' for review versions, 'empty' for camera ready
\newcommand\vldbpagestyle{plain} 

\begin{document}
\title{Active learning isn't enough: Towards interpretable implicit recommendation for Exploratory Data Analysis}

%%
%% The "author" command and its associated commands are used to define the authors and their affiliations.
\author{David Adams}
\affiliation{%
  \institution{University of Melbourne}
  \streetaddress{Grattan Street Parkville}
  \city{Melbourne}
  \state{Victoria}
  \postcode{3010}
}
\email{david.adams1@student.unimelb.edu.au}

%%
%% The abstract is a short summary of the work to be presented in the
%% article.
\begin{abstract}
Exploratory Data Analysis (EDA) is a critical step in the data science workflow, enabling analysts to uncover patterns, spot anomalies, and test hypotheses. However, traditional EDA techniques often fall short in providing interpretable insights, particularly when dealing with complex datasets. In this paper, we propose a novel approach that combines active learning with interpretable implicit recommendation systems to enhance the EDA process. Our method leverages user feedback to refine recommendations, ensuring that the insights generated are not only relevant but also easily interpretable by analysts. We demonstrate the effectiveness of our approach through a series of experiments on real-world datasets, highlighting its potential to transform the way data analysts interact with and understand their data.
\end{abstract}

\maketitle

%%% do not modify the following VLDB block %%
%%% VLDB block start %%%
\pagestyle{\vldbpagestyle}
\begingroup\small\noindent\raggedright\textbf{PVLDB Reference Format:}\\
\vldbauthors. \vldbtitle. PVLDB, \vldbvolume(\vldbissue): \vldbpages, \vldbyear.\\
\href{https://doi.org/\vldbdoi}{doi:\vldbdoi}
\endgroup
\begingroup
\renewcommand\thefootnote{}\footnote{\noindent
This work is licensed under the Creative Commons BY-NC-ND 4.0 International License. Visit \url{https://creativecommons.org/licenses/by-nc-nd/4.0/} to view a copy of this license. For any use beyond those covered by this license, obtain permission by emailing \href{mailto:info@vldb.org}{info@vldb.org}. Copyright is held by the owner/author(s). Publication rights licensed to the VLDB Endowment. \\
\raggedright Proceedings of the VLDB Endowment, Vol. \vldbvolume, No. \vldbissue\ %
ISSN 2150-8097. \\
\href{https://doi.org/\vldbdoi}{doi:\vldbdoi} \\
}\addtocounter{footnote}{-1}\endgroup
%%% VLDB block end %%%

%%% do not modify the following VLDB block %%
%%% VLDB block start %%%
\ifdefempty{\vldbavailabilityurl}{}{
\vspace{.3cm}
\begingroup\small\noindent\raggedright\textbf{PVLDB Artifact Availability:}\\
The source code, data, and/or other artifacts have been made available at \url{\vldbavailabilityurl}.
\endgroup
}
%%% VLDB block end %%%

\section{Introduction}

No one can really agree on a definition of EDA. It is generally accepted that EDA is a process of exploring data to find patterns, anomalies, and relationships. However, the specific techniques and tools used for EDA can vary widely depending on the context and the goals of the analysis.

% what is an insight

But more importantly no one can agree on what an Insight is. Is it a correlation? A trend? A cluster? A outlier? A distribution? A summary statistic? A visualization? A hypothesis? A question? A story? Some people attempt to provide an informal mathematical definition \cite{PDFExtractingTopK} but these are usually inefficient

% active learning is not enough

The most sucessful interactive reccomendation systems such as AIDE \cite{dimitriadouAIDEActiveLearningBased2016}

% One interestingness measure is not enough




\section{Query results alone are not enough}


\section{Citations}


\begin{acks}
\end{acks}

%\clearpage

\bibliographystyle{ACM-Reference-Format}
\bibliography{references}

\end{document}
\endinput
